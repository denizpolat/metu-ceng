\documentclass[12pt,a4paper]{article}
\usepackage[utf8]{inputenc}
\usepackage{float}
\usepackage{amsmath}
\usepackage{graphicx} % better \includegraphics
\usepackage{enumerate}
\usepackage{lipsum} % placeholder text
\usepackage{csquotes} % \enquote for automatic quotations

% You can bring in additional packages if you need to

\topmargin=-2cm

\begin{document}

\section*{Student Information} 
% Write your full name and id number between the colon and newline
% Put one empty space character after colon and before newline
Full Name: \\
Id Number: 
% Write your answers below the section tags

\section*{HTTP \& DNS (70 Points)}

Type your answers under the appropriate subsections.

\subsection*{1. (8 Points)}

\subsection*{2. (10 Points)}

\subsection*{2. (Bonus) (10 Bonus Points)}

\subsection*{3. (15 Points)}

\subsection*{4. (15 Points)}

\subsection*{5a. (7 Points)}

\subsection*{5b. (15 Points)}

% Including figures
% example:
\begin{figure}[htbp] % placement specifiers, https://en.wikibooks.org/wiki/LaTeX/Floats,_Figures_and_Captions#Figures
    % 50% of the text width, you can play around with that 
    \centerline{\includegraphics[width=0.5\textwidth]{720x480.png}} 
    \caption{A caption that summarizes the figure}%
    \label{fig:example_figure} % 
\end{figure}

You can then reference your figure like so; the answer is derived from Figure~\ref{fig:example_figure}. The reference will update itself automatically to always refer to the correct figure. Note that you still have to manually write \enquote{Figure}.


\section*{HTTPS \& TLS (30 Points)}

\subsection*{1. (10 Points)}

\subsection*{2. (10 Points)}

\subsection*{3. (10 Points)}

% Please do not submit the tex file, pdf file is enough.
% Hope you have fun :) 
\end{document}
